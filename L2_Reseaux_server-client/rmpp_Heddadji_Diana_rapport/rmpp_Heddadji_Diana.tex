\documentclass[handout]{beamer}

%LES PAQUETS
\usepackage[frenchb]{babel}
\usepackage[T1]{fontenc}
\usepackage[utf8x]{inputenc}
\usepackage {multicol}

%LE THEME
\usetheme{Warsaw}
\definecolor{bazaar}{rgb}{0.4, 0.47, 0.78}
\usecolortheme[named=bazaar]{structure}
\title[Réseaux : modèles, protocoles, programmation]{\Huge{Projet Chat}}
\subtitle{V. Python}
\author{Diana Heddadji}
\institute{PARIS VIII - UFR MITSIC / L2 Informatique}


\begin{document}
	%Présentation
	\begin{frame}
	\titlepage
	\includegraphics{Images/logop8.png}
	\end{frame}

	%Sommaire
  	\begin{frame}
  	\frametitle{Sommaire}
	\begin{multicols}{2}
	\large {\hyperlink{I}{\\{Introduction}}}
			\newline \\
		\begin{itemize}
			\setbeamertemplate{itemize item}[ball]
			\item\small{Choix du projet de groupe}
			\setbeamertemplate{itemize item}[ball]
			\item \small{Protocole}
			\newline \\
		\end{itemize}

	\large {\hyperlink{O}{\\{Organisation du Projet}}}
			\newline \\
		\begin{itemize}
			\setbeamertemplate{itemize item}[ball]
			\item \small{Les fichiers}
		\end{itemize}

	\large {\hyperlink{A}{\\{Une Approche vers le Code}}}
			\newline \\
		\begin{itemize}
			\setbeamertemplate{itemize item}[ball]
			\item \small{Le Client et le Serveur}
				\begin{itemize}
				\setbeamertemplate{itemize item}[ball]
				\item \small{Importation du socket}
				\setbeamertemplate{itemize item}[ball]
				\item \small{Connexion du socket}
				\setbeamertemplate{itemize item}[ball]
				\item \small{Faire écouter le socket}
				\setbeamertemplate{itemize item}[ball]
				\item \small{Accepter une connexion venant du client}
				\setbeamertemplate{itemize item}[ball]
				\item \small{Création du Client}
				\setbeamertemplate{itemize item}[ball]
				\item \small{Les méthodes send et recv}
				\setbeamertemplate{itemize item}[ball]
				\item \small{La méthode select}
				\setbeamertemplate{itemize item}[ball]
				\item \small{Fermer la connexion}
				\end{itemize}
		\end{itemize}
  	\tableofcontents[currentsection, hideothersubsections, pausesubsections]
	\end{multicols}
  	\end{frame}
	
	%Introduction | Choix du projet de groupe
	\begin{frame} [label=I]
	\frametitle{Introduction | Choix du projet de groupe}
	{\itshape Le but de ce projet est de parvenir à réaliser un chat en groupe de 	cinq.\newline \\Chaque membre du groupe choisira un langage de programmation 		différent afin 	de créer un client et un serveur en suivant un protocole 			précis.}
  	\end{frame}

	%Introduction | Protocole
	\begin{frame}
	\frametitle{Introduction | Protocole}
		\href{protocole.md}{protocole.md : Protocole}
	\end{frame}
	
	%Organisation du Projet | Les fichiers
	\begin{frame} [label=O]
	\frametitle{Organisation du Projet | Les fichiers}
	Dans la partie python du projet, il y a deux fichiers :
	\begin{itemize}
		\setbeamertemplate{itemize item}[ball]
		\item \bfseries{client.py}
		\setbeamertemplate{itemize item}[ball]
		\item \bfseries{serveur.py}
	\end{itemize}
	\end{frame}
	
	%Une Approche vers le Code | Importation du socket
	\begin{frame} [label=A]
	\frametitle{Une Approche vers le Code | Importation du socket}
	On commence par importer notre socket avec :\newline \\
	{\bfseries import socket}\newline \\
	Pour créer notre socket, il est nécessaire de faire un appel au constructeur 				{\bfseries socket}.\newline \\
	On aura besoin des deux paramètres suivants s'il s'agit d'une connexion {\bfseries 			TCP} :
	\begin{itemize}
		\setbeamertemplate{itemize item}[ball]
		\item {\bfseries socket.AF-INET} : la familles des adresses internet;
		\setbeamertemplate{itemize item}[ball]
		\item {\bfseries socket.SOCK-STREAM} : Pour le protocole {\bfseries TCP}, 					{\bfseries SOCK-STREAM} est le type du {\bfseries socket}.\newline
	\end{itemize}
 	{\bfseries main-connection = socket.socket(socket.AF-INET, socket.SOCK-STREAM)}
	\end{frame}

	%Une Approche vers le Code | Connexion du socket
	\begin{frame}
	\frametitle{Une Approche vers le Code | Connexion du socket}
	Pour connecter le {\bfseries socket}, on utilisera la méthode {\bfseries bind} dans le 	cas où le serveur attend des clients.
	Cette dernière prend le tuple {\bfseries (nom-hote, port)}.\newline \\
	{\bfseries main-connection.bind(('', 2222))}
	\end{frame}
	
	%Une Approche vers le Code | Faire écouter le socket
	\begin{frame}
	\frametitle{Une Approche vers le Code | Faire écouter le socket}
	Pour faire écouter notre {\bfseries socket}, il faut lui renseigner le nombre maximum 		de connexions qu'il peut recevoir sur le port sans les accepter.
	Ceci ce fait par la méthode {\bfseries listen} qui prend ce nombre en paramètre.
	\end{frame}
	
	%Une Approche vers le Code | Accepter une connexion venant du client
	\begin{frame}
	\frametitle{Une Approche vers le Code | Accepter une connexion venant du client}
	Pour accepter une connexion venant du client, l'usage de la méthode {\bfseries accept} 	est nécessaire afin de bloquer le programme tant que personne ne s'est connecté.			\newline \\
	Cette méthode retourne deux informations :
	\begin{itemize}
		\setbeamertemplate{itemize item}[ball]
		\item La première est le {\bfseries socket} connecté qui vient de se créer qui 				nous permettra par la suite de dialoguer avec notre client.
		\setbeamertemplate{itemize item}[ball]
		\item La seconde est un tuple comportant l'adresse {\bfseries IP} ainsi que le 				port de connexion du client.\newline \\
	\end{itemize}
	{\bfseries connection-client, connection-info = main-connection.accept()}
	\end{frame}

	%Une Approche vers le Code | Création du Client Part. 1
	\begin{frame}
	\frametitle{Une Approche vers le Code | Création du Client Part. 1}
	Pour créer ensuite le client, le principe est le même que celui du {\bfseries socket} 		:\newline \\{\bfseries import socket\\
	connection-server = socket.socket(socket.AF-INET, socket.SOCK-STREAM)}						\newline \\
	On connecte ensuite le client avec la méthode {\bfseries connect}. Celle-ci prend un 		tuple en paramètre comportant le nom d'hote et le numéro du port identifiant le 			serveur auquel on veut se connecter.\newline \\
	{\bfseries connection-server.connect(('localhost', 2222))}\newline \\
	Le serveur et le client étant désormais connectés, cela signifie que la méthode 			{\bfseries accept} ne bloque plus le programme étant donné qu'elle vient d'accepter 		la connexion du client.
	\end{frame}

	%Une Approche vers le Code | Création du Client Part. 2
	\begin{frame}
	\frametitle{Une Approche vers le Code | Création du Client Part. 2}
	On peut afficher l'adresse {\bfseries IP} et le {\bfseries port} du client avec {\bfseries 	print(connection-info)} du côté du serveur.\newline \\
	L'adresse IP vaut {\bfseries 127.0.0.1} il s'agit de l'adresse {\bfseries IP local} de la machine 		est donc {\bfseries localhost} redirige vers cette adresse {\bfseries IP}.\newline \\
	Pour faire communiquer nos sockets, il suffit d'utiliser les méthodes {\bfseries send} 		et {\bfseries recv}.\newline \\
	Au niveau du serveur cela se fera comme suit :\newline \\
	{\bfseries client.send(b"ok")}
	\begin{itemize}
		\setbeamertemplate{itemize item}[ball]
		\item {\bfseries send} : retourne le nombre de caractères envoyés
		\setbeamertemplate{itemize item}[ball]
		\item {\bfseries recv} : prend ce que retourne {\bfseries send} c'est-à-dire, le 			nombre de caractères à lire.
	\end{itemize}	
	Côté client, on réceptionnera le message que l'on vient d'envoyer.
	\end{frame}	


	%Une Approche vers le Code | Création du Client Part. 3
	\begin{frame}
	\frametitle{Une Approche vers le Code | Création du Client Part. 3}
	Étant donné qu'on ne sait pas à l'avance le nombre de caractères qu'on recevra, on lui donne conventionnellement la valeur 2000.\\ Si le message comporte plus de caractères 		alors on récupère le reste après.\newline \\
	Côté client, ceci se fera comme suit :\\
	{\bfseries 	msg-received  = connection-server.recv(2000)}
	\end{frame}

	%Une Approche vers le Code | Les méthodes send et recv
	\begin{frame}
	\frametitle{Une Approche vers le Code | Les méthodes encode et decode}
	À chaque fois que le serveur reçoit un message, il envoie un accusé de réception : {\bfseries "OK"}.	\newline \\
	Côté client, on peut remarquer l'utilisation des méthodes de {\bfseries str} :				{\bfseries encode} et {\bfseries decode};\newline \\
	{\bfseries encode} sert à partir d'un nom d'encodage {\bfseries (Utf-8)} de passer un 		{\bfseries str} en chaîne {\bfseries bytes} et {\bfseries decode} permet de faire 					exactement l'inverse.\newline \\
	Les informations transmises par {\bfseries send} et {\bfseries recv} sont des chaînes 		de {\bfseries bytes}, pas des {\bfseries str} et c'est {\bfseries encode} et 				{\bfseries decode} qui permettent de traduire les messages reçus et envoyés.
	\end{frame}

	%Une Approche vers le Code | La méthode select Part. 1
	\begin{frame}
	\frametitle{Une Approche vers le Code | La méthode select Part. 1}
	Du côté de notre serveur, on peut voir la présence du module {\bfseries select}, 			celui-ci va interroger plusieurs clients dans l'attente d'un message à réceptionner, 		sans mettre en pause le programme.\newline \\

	{\bfseries Select} va donc écouter sur une liste de clients et retourner au bout d'un 		certain temps.\newline \\
	Ce qui est retourné est la liste des clients qui ont un message à réceptionner.				\newline \\
	On utilisera la méthode {\bfseries select} qui prend trois ou quatre arguments et en 		retourne trois.
	\end{frame}
	
	%Une Approche vers le Code | La méthode select Part. 2
	\begin{frame}
	\frametitle{Une Approche vers le Code | La méthode select Part. 2}
	Dans l'ordre, les paramètres de {\bfseries select} sont : \newline \\
	\begin{itemize}
		\setbeamertemplate{itemize item}[ball]
		\item {\bfseries rlist} : la liste des sockets en attente d'être lus;
		\setbeamertemplate{itemize item}[ball]
		\item {\bfseries wlist} : la liste des sockets en attente d'être écrits;
		\setbeamertemplate{itemize item}[ball]
		\item {\bfseries xlist} : la liste des sockets en attente d'une erreur;
		\setbeamertemplate{itemize item}[ball]
		\item {\bfseries timeout} : le délai pendant lequel la fonction attend avant de 		retourner.
	\end{itemize}
	\end{frame}

	%Une Approche vers le Code | La méthode select Part. 3
	\begin{frame}
	\frametitle{Une Approche vers le Code | | La méthode select Part. 3}
	On créer une liste {\bfseries connected-client} afin d'y mettre des 			{\bfseries sockets} de façon à ce que {\bfseries select} les surveille et 		puisse retourner dès qu'un {\bfseries socket} est prêt à être lu.
	Ainsi, le programme ne bloquera pas et pourra recevoir des messages de 			plusieurs clients.\newline \\
	
	En précisant notre {\bfseries timeout}, {\bfseries select} retournera au bout du temps en secondes que l'on aura indiqué {\bfseries (ici 0.05)}, ou si un {\bfseries socket} est prêt à être lu.
	\end{frame}

	%Une Approche vers le Code | La méthode select Part. 4
	\begin{frame}
	\frametitle{Une Approche vers le Code | | La méthode select Part. 4}
	{\bfseries select} renvoie trois listes {\bfseries rlist}, {\bfseries wlist} 		et {\bfseries xlist}.\newline \\

	{\bfseries read-client, list1, list2 = select.select(connected-client, [], 			[],	0.05)}\newline \\
	Cette instruction va écouter les sockets contenus dans 	la liste {\bfseries 		connected-client}.
	\end{frame}
	
	%Une Approche vers le Code | Fermer la connexion
	\begin{frame}
	\frametitle{Une Approche vers le Code | Fermer la connexion}
	Enfin pour fermer la connexion, il suffit d'appeler la méthode {\bfseries close} sur 		le {\bfseries socket}.\newline \\
	Côté client :\newline \\
	{\bfseries connection-server.close()}\newline \\
	Côté serveur:\newline \\
	{\bfseries main-connection.close()}\newline \\
	Le serveur tourne jusqu'à recevoir le message {\bfseries "BYE"}.
	\end{frame}
	
\end{document}